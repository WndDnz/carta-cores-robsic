\documentclass[a4paper, foldmark, notumble]{leaflet}
\usepackage[utf8]{inputenc}
\usepackage[T1]{fontenc}
\usepackage{libertine}
\usepackage{tikz}
\renewcommand*\familydefault{\sfdefault}
\usepackage{lipsum}

\definecolor{primary}{HTML}{6CC24A}
\definecolor{secondary}{HTML}{6E7CA0}
\definecolor{accent}{HTML}{003A70}
\definecolor{alert}{HTML}{DA291C}

\title{Carta de Cores Oficiais do RobSIC}


\AddToBackground{1}{\tikz \fill [primary] (0,0) rectangle (4,10);}

\begin{document}

\maketitle

\lipsum[1-2]

\pagebreak

\section{Cor Principal}
\begin{center}

  \tikz \fill [primary] (0,0) rectangle (4,4);
  \vskip3mm
  {\large\bfseries Pantone 360 C}\\
  {\large\bfseries RGB \#6cc24a}
\end{center}

\pagebreak

\section{Cor Secundária}
\begin{center}

  \tikz \fill [secondary] (0,0) rectangle (4,4);
  \vskip3mm
  {\large\bfseries Pantone 360 C}\\
  {\large\bfseries RGB \#6E7CA0}
\end{center}

\pagebreak

\section{Cor de Destaque}
\begin{center}

  \tikz \fill [accent] (0,0) rectangle (4,4);
  \vskip3mm
  {\large\bfseries Pantone 360 C}\\
  {\large\bfseries RGB \#003A70}
\end{center}

\pagebreak

\section{Cor de Alerta}
\begin{center}

  \tikz \fill [alert] (0,0) rectangle (4,4);
  \vskip3mm
  {\large\bfseries Pantone 360 C}\\
  {\large\bfseries RGB \#DA291C}
\end{center}


\end{document}
